% !TEX root = oa-sampling.tex
%%%%%%%%%%%%%%%%%%%%%%%%%%%%%%%%%%%%%%%%%%%%%%%%%%%%%%%%%%%%%%%%%%%%%%%%%
% Additional packages
%%%%%%%%%%%%%%%%%%%%%%%%%%%%%%%%%%%%%%%%%%%%%%%%%%%%%%%%%%%%%%%%%%%%%%%%%
% extra options for table
% remove this if thing looks different
% \usepackage{colortbl}
\usepackage{xcolor}
\usepackage{xfrac}
% \usepackage{float}

\usepackage{paralist}
\usepackage{url}
% \usepackage[usenames,dvipsnames]{xcolor}
\hypersetup{
  % colorlinks=true,
  % linkcolor=black,
  % citecolor=black,
  % urlcolor=black,
  breaklinks=true
}
% \hypersetup{draft}% include if you get errors about a link spanning page breaks

\usepackage[sharp]{easylist}
\let\proof\relax
\let\endproof\relax
\usepackage{amsthm}
\usepackage{mathtools}
\usepackage{bm}
% \usepackage{pdfpages}
\usepackage{wrapfig}
\usepackage{adjustbox}
% \usepackage{subfig}
% \usepackage{subcaption}
\usepackage{units}
\usepackage{siunitx}
%\sisetup{output-exponent-marker=\ensuremath{\mathrm{e}}}
\sisetup{exponent-product=\!\!\times\!\!}
\usepackage{soul}
\usepackage{array}
% \usepackage{calc}
\usepackage{pgf}
\usepackage{tikz}
\usepackage{xspace}
% \usepackage[percent]{overpic}
\usepackage{enumitem}
% \usepackage{afterpage}
\usepackage[normalem]{ulem}  % for strike-through (\sout)
% \usepackage{multirow}
% \usepackage[makeroom]{cancel}
% \usepackage{bm}
\usepackage[utf8]{inputenc}
% % for commenting
% \usepackage[final]{pdfcomment}  % for Final draft
\usepackage[outline]{contour}
%\contourlength{0.5em}
% \usepackage{tcolorbox}
\usepackage{datatool} % for reading CSV files
\usepackage{readarray}

% \usepackage{inconsolata}
% \usepackage[scaled=0.9]{FiraMono}

\usepackage{listings}
%\usepackage[capitalize]{cleveref}

%New colors defined below
\definecolor{codegreen}{rgb}{0,0.5,0}
\definecolor{codegray}{rgb}{0.5,0.5,0.5}
\definecolor{codepurple}{rgb}{0.58,0,0.82}
\definecolor{codered}{rgb}{0.9,0,0}
% \definecolor{codeblue}{rgb}{0.13,0.13,1}
\definecolor{codeblue}{rgb}{0.15,0.2,.83}
% \definecolor{otherkeywords}{rgb}{0.07,0.22,.48}

%Code listing style named "mystyle"
\makeatletter
\lstdefinestyle{mystyle}{
captionpos=t,
keepspaces=true,
numbers=left,
numbersep=1em,
tabsize=4,
showspaces=false,
showtabs=false,
breaklines=true,
frame=lines,
showstringspaces=false,
breakatwhitespace=true,
escapeinside={(*@}{@*)},
commentstyle=\color{codegreen},
keywordstyle=\color{codeblue}\bfseries,
stringstyle=\color{codered},
numberstyle=\ttfamily\fontsize{6.65}{7.5}\selectfont\color{codegray}\bfseries,
basicstyle=\ttfamily\lst@ifdisplaystyle\fontsize{6.65}{7.5}\else\fontsize{8.65}{9.5}\fi\selectfont
}

%"mystyle" code listing set
\lstset{style=mystyle}
\lstset{
emph={randfloat,permute,boseOA,bushOA,offset,pow,evalPoly,toBaseS,allBut1,allButJ},
emphstyle=\ttfamily\lst@ifdisplaystyle\fontsize{6.65}{7.5}\else\fontsize{8.65}{9.5}\fi\selectfont\color{codepurple}
}
\makeatother


% \def\inline{\lstinline[basicstyle=\ttfamily\fontsize{8.65}{9.5}\selectfont]}

% \lstset{emph={vector},emphstyle=\ttfamily\fontsize{6.65}{7.5}\selectfont\color{codeblue}\bfseries}


\newtheorem{mydef}{Definition}

%Comments
% Select the type of commenting you want to use: none, pdf margin comments, or in-text comments
% \newcommand{\COMMENT}[3]{}  % for Final draft, removes comments
% \newcommand{\COMMENT}[3]{\pdfcomment[avatar=#1]{#1: #3}}  % For work in progress using PDF annotations
\newcommand{\COMMENT}[3]{\textcolor{#2}{{\footnotesize[\textsc{#1}: #3}]}}  % For work in progress using in-text comments

%\titlespacing*{\paragraph}{0pt}{0.2\baselineskip}{\baselineskip}

\newcommand{\TODO}[1]{\COMMENT{TODO}{red}{#1}}
\newcommand{\WJ}[1]{\COMMENT{WJ}{blue}{#1}}
\renewcommand{\AE}[1]{\COMMENT{AE}{orange}{#1}}
\newcommand{\PC}[1]{\COMMENT{PC}{teal}{#1}}
\newcommand{\AK}[1]{\COMMENT{AK}{olive}{#1}}
\newcommand{\CK}[1]{\COMMENT{CK}{purple}{#1}}

\newcommand{\IGNORE}[1]{}

\definecolor{DarkGreen}{rgb}{0.0,0.6,0.2}
\newcommand{\REMOVE}[1]{{\color{red}\sout{#1}}} % for work in progress
\newcommand{\ADD}[1]{{\textcolor{DarkGreen}{#1}}} % for work in progress
% \newcommand{\ADD}[1]{#1} % for work in progress
\newcommand{\REPLACE}[2]{\ADD{#2}} % for work in progress, showing the old text


% \definecolor{CancelColor}{rgb}{0.7,0.7,0.7}
% \renewcommand{\CancelColor}{\color{CancelColor}}

% \crefname{lstlisting}{listing}{listings}
% \Crefname{lstlisting}{Listing}{Listings}
% \crefname{lstnumber}{line}{line}
% \crefname{lstnumber}{Line}{Line}
\newcommand{\crefrangeconjunction}{--}

%\crefname{chapter}{Chapter}{Chapters}
%\crefname{section}{Sec.}{Secs.}
%\crefname{subsection}{Sec.}{Secs.}
%\crefname{subsubsection}{Sec.}{Secs.}
%\crefname{figure}{Fig.}{Figs.}
%\crefname{table}{Table}{Tables}
%\crefname{listing}{Listing}{Listings}

%%Math
\newcommand{\dif}{\mathrm{d}}

\newcommand{\OA}[1]{OA$(#1)$}
\newcommand{\sIndex}{\lambda}
\newcommand{\sStrength}{t}
\newcommand{\sDimension}{d}
\newcommand{\sNumSamples}{N}
\newcommand{\sLevels}{s}
\newcommand{\bigO}{\mathcal{O}}

\let\oldabs\abs % Store original \abs as \oldabs
\let\abs\undefined % "Undefine" \abs
\DeclarePairedDelimiter\abs{\lvert}{\rvert}
\DeclarePairedDelimiter\ceil{\lceil}{\rceil}
\DeclarePairedDelimiter\floor{\lfloor}{\rfloor}

\newcommand\dropshadowlabelcolor[5]{
  \node at (#3, #4) {\fontsize{9pt}{2pt}\selectfont\contour{#2}{\color{#1}#5}};
  \node at (#3, #4) {\fontsize{9pt}{2pt}\selectfont\color{#1}#5};
}



% Wojciech Jarosz:
% Don't use small caps for author names in citations, only in the bibliography
\providecommand{\mkbibnamefamily}[1]{#1}
\newcommand\lastnameformat[1]{\textsc{#1}}
\newcommand\givennameformat[1]{\textsc{#1}}
% \renewcommand*{\mkbibnamefamily}[1]{\textsc{#1}}
% \renewcommand*{\mkbibnamegiven}[1]{\textsc{#1}}
%\AtBeginBibliography{%
  %\renewcommand\mkbibnamelast{\lastnameformat}%
  %\renewcommand\mkbibnamefamily{\lastnameformat}%
  %\renewcommand{\mkbibnamegiven}{\givennameformat}
%}

% use sentence case
% from: https://tex.stackexchange.com/questions/22980/sentence-case-for-titles-in-biblatex
%\DeclareFieldFormat{sentencecase}{\MakeSentenceCase{#1}}

%\renewbibmacro*{title}{%
  %\ifthenelse{\iffieldundef{title}\AND\iffieldundef{subtitle}}
    %{}
    %{\ifthenelse{\ifentrytype{article}\OR\ifentrytype{inbook}%
      %\OR\ifentrytype{incollection}\OR\ifentrytype{inproceedings}%
      %\OR\ifentrytype{inreference}}
      %{\printtext[title]{%
        %\printfield[sentencecase]{title}%
        %\setunit{\subtitlepunct}%
        %\printfield[sentencecase]{subtitle}}}%
      %{\printtext[title]{%
        %\printfield[titlecase]{title}%
        %\setunit{\subtitlepunct}%
        %\printfield[titlecase]{subtitle}}}%
     %\newunit}%
  %\printfield{titleaddon}}

% End biblatex tweaks
